\usepackage{geometry}
%\pagenumbering{gobble}
\usepackage{fontspec}
\usepackage[english,french,ecclesiasticlatin.usej,activeacute]{babel}
     \babelprovide[hyphenrules=latin]{ecclesiasticlatin}
\usepackage[compact,nobottomtitles]{titlesec}
\usepackage{fancyhdr}
\usepackage{needspace}
\usepackage{lettrine}
\usepackage[autocompile]{gregoriotex}
\usepackage{graphicx}
\usepackage{hyperref}
\usepackage{refcount}

\AtBeginDocument{\setlength{\parindent}{1em}} %% should set 1em to EB Garamond not Computer Modern.

%% General scale of all graphical elements. Also borrowed from MB
%% Values different from 1 are largely untested per MB
%% Used in those commands (e.g. everything FontSpec) that use a scale parameter.
\newcommand{\customscale}{1}

\makeatletter
\newcommand{\YEAR}{\@Roman{\the\year}}
\makeatother

\geometry{letterpaper}
\geometry{bindingoffset=0.5in,
inner=1in,
outer=1.35in,
top=1.25in,
bottom=1.25in,
headsep=0.75in
} %%

\setmainfont{EB Garamond}[UprightFont=EB Garamond Regular,
ItalicFont= EB Garamond Italic,
BoldFont= EB Garamond Bold,
Ligatures=Rare,
Numbers=Proportional,
Numbers=OldStyle]

\NewDocumentCommand\capspace{m}{{\addfontfeature{LetterSpace=5.0}{#1}}}
\NewDocumentCommand\scspace{m}{\textsc{{{\addfontfeature{LetterSpace=5.0}{#1}}}}}

%% should prevent flat turning custos into a clef, as Solesmes  uses an altered custos in only one chant per Élie Roux.
\gresetcustosalteration{invisible}

%produces a score with a smaller initial (default is 40 pt) and annotations like in the Solesmes books; the 1st is optional and can be omitted as needed.
 \NewDocumentCommand{\gscore}{O{}mm}{%
 \greannotation{#1}%
\greannotation{#2}%
\grechangestyle{initial}{\fontsize{28}{28}\selectfont}%
 \gregorioscore{partitions/#3}}
 

  \NewDocumentCommand{\initialscore}{O{}m}{%
      \grechangedim{initialraise}{0.5cm}{scalable}%
  \greannotation{#1}%
   \grechangestyle{initial}{\fontspec{EB Garamond Initials}\fontsize{45}{45}\selectfont}%
 \gregorioscore{partitions/#2}
   \grechangedim{initialraise}{0cm}{scalable}}

%% score with no initial, e.g psalms, common tones, etc.
\newcommand{\smallscore}[2][y]{
  \gresetinitiallines{0}
  \gregorioscore{partitions/#2} %% à remplacer avec NewDocumentCommand ; les arguments ne marchent pas correctement … que fait le [y] ?
  \gresetinitiallines{1}
}

%% sets * and † to font called via fontspec
\def\GreStar{*}
\def\GreDagger{†}

\gresetspecial{V/}{{\fontspec[Scale=\customscale]{EB Garamond}℣.~}}
\gresetspecial{R/}{{\fontspec[Scale=\customscale]{EB Garamond}℟.~}}
\gresetspecial{+}{{\fontspec[Scale=\customscale]{EB Garamond}†~}}
\gresetspecial{*}{\gresixstar}
\gresetspecial{cross}{\cc}

%% fine-tuning the behavior of text placed under bars. We use the so-called "new algorithm" which
%% places the bar in the middle of surrounding notes, and the text in the middle of surrounding text.
%% however, we restrict drastically the deviation of the text from the position of the bar.
\grechangedim{maxbaroffsettextleft}{0.5mm}{scalable}
\grechangedim{maxbaroffsettextright}{0.5mm}{scalable}

\NewDocumentCommand{\zerobaroffsettextleft}{}%
{%
\grechangedim{maxbaroffsettextleft}%
{0cm}%
{scalable}%
}

\NewDocumentCommand{\zerobaroffsettextright}{}%
{%
\grechangedim{maxbaroffsettextright}%
{0cm}%
{scalable}%
}

%%%the default for left bar offset is 0.3cm. Right is 0.15cm. This needs to be applied after a score where the bar offset is modified.
\NewDocumentCommand{\resetbaroffsettextleft}{}%
{%
\grechangedim{maxbaroffsettextleft}%
{0.5mm}%
{scalable}%
}

\NewDocumentCommand{\resetbaroffsettextright}{}%
{%
\grechangedim{maxbaroffsettextright}%
{0.5mm}%
{scalable}%
}

\NewDocumentCommand{\bigtitle}{m}{
\needspace{5\baselineskip}  %% very small if there are neumes above the staff, including flats in mode 2, e.g. O Doctor optime
\vspace{\baselineskip}
 {\centering{\large#1}\par}
}

\NewDocumentCommand{\smalltitle}{m}{
\needspace{5\baselineskip}  %% very small if there are neumes above the staff, including flats in mode 2, e.g. O Doctor optime
\vspace{\baselineskip}
 {\centering #1\par}
}


\NewDocumentCommand{\rubrique}{m}{%
    {%
        \fontsize{10}{12}%
        \selectfont%
        \textit{%
        %
        #1%
    }%
    }}
    
%% macro to print normal text inside of rubric (name of a chant or prayer, etc.)

\NewDocumentCommand{\normaltext}{m}{%
    {%
        \normalfont%
        #1%
    }%
    }
    
    \NewDocumentCommand{\bigrule}{}{%
    \par%
    {%
        \centering%
        \rule{0.6\textwidth}{0.4pt}%
        \par%
    }% typesets a horizontal rule like on the page with the prayers before and after the office.
}  %%If you're using color at all in your document, you might want to either force \myrule to use black or make it cusotmizable. (from u/Independent-Comb-257)


\pagestyle{fancy}
\fancyhead{}
\fancyfoot{}
\renewcommand\headrulewidth{0.4pt}

\fancyhead[RO]{\small{\addfontfeature{LetterSpace=5.0}\rightmark}\hspace{1cm}\thepage}
\fancyhead[LE]{\small\thepage\hspace{1cm}{\addfontfeature{LetterSpace=5.0}LAUS IN ECCLESIA : CHANTS}}

\titleformat{\section}[display]{\huge\filcenter\sc\addfontfeature{LetterSpace=5.0}}{}{0pt}{} 
\titleformat{\subsection}[display]{\large\scshape\filcenter\addfontfeature{LetterSpace=5.0}}{}{}{}
\setcounter{secnumdepth}{0}
\titlespacing{\section}{0pt}{*0}{*0}

% \NewDocumentCommand{\mysection}{m}{\section{#1}\setheaders{{\scshape\addfontfeature{LetterSpace=5.0}#1}}{{\scshape\addfontfeature{LetterSpace=5.0} #1}}}
 
\NewDocumentCommand{\header}{m}{\setheaders{{\scshape\addfontfeature{LetterSpace=5.0}#1}}{{\scshape\addfontfeature{LetterSpace=5.0} #1}}}

\addto\captionsecclesiasticlatin{\renewcommand\contentsname{\centerline{Table of Contents. Table de matières.}}}