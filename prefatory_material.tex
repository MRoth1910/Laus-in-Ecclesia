\begin{titlepage}
 \begin{center}
 \fontsize{40}{50}\selectfont\addfontfeature{LetterSpace=5.0}{\MakeUppercase{Laus in Ecclesia}}
 \end{center}
  \vspace{\baselineskip}
\centering
%\includegraphics[width=10cm,height=10cm,keepaspectratio]{Annunciatie_Leven_van_Maria_(serietitel),_RP-P-OB-1406}
\includegraphics[scale=0.20,keepaspectratio]{Annunciatie_Leven_van_Maria_(serietitel),_RP-P-OB-1406}
\begin{center}
 \fontsize{32}{40}\selectfont\addfontfeature{LetterSpace=5.0}{\MakeUppercase{Chants}}
\end{center}

 \vfill
 
{\centering{\LARGE\scspace{sancta maria ab assumptione}\par}}
 {\centering{\capspace{\YEAR}\par}}
 \clearpage\thispagestyle{empty}
\end{titlepage}

\frontmatter

\begin{otherlanguage}{english}
\lettrine{T}{his} book contains the scores necessary to complete the \textit{Laus in Ecclesia} course in Gregorian chant, developed by the \textit{Schola Saint-Grégoire} in France, adapted in its current form by the monks of\ Triors, and translated into English by their brothers at Clear Creek in Oklahoma, in the United States. Among the exercises are firstly singing exercises with audio meant to guide the student, as the course remains fundamentally a correspondance course meant to be done independently. Other exercises are provided without audio so that the student can learn unfamiliar chants like one does preparing for the Holy Mass on Sundays.

There are also written exercises to be completed and sent via mail or electronically according to the rules for the course. Although the written exercises do not necessarily require a complete score but only a portion of it, providing the full score allows the student to further learn the entire chant or to do so in a group setting if several students work together, with or without an instructor sufficiently competent in chanting according to the method of So\-lesmes to guide the students.

The student still needs a copy of the textbook for Level I. However, this course book replaces a copy of the \textit{Graduale Romanum} or the \textit{Liber Usualis} or for the student to find copies of the scores independently via the internet. However, as this book only includes the scores, the student still needs to draw tables and to copy by hand or electronically parts of scores according to the directions for some of the written exercises, which are in the textbook.\end{otherlanguage}

\begin{otherlanguage}{french}
\lettrine{C}{e} livre contient les partitons nécessaires de poursuivre la formation en chant grégorien \textit{Laus in Ecclesia,} dévéloppée dans un premier temps par le Schola Saint-Grégoire du Mans, puis adaptée sous la forme actuelle par les moines de l'abbaye de Triors. Une traduction anglaise a été réalisée par leurs confrères de l'abbaye de Clear Creek dans l'Oklahoma (États-Unis). Parmi les exercices, il y a d'abord des exercices de chant, avec des enregistrements qui guident l'élève, puisque le cours reste fondamentalement un cours par correspondance que poursuit indépendamment l'élève. On donne d'autres exercices semblables sans l'audio pour que l'élève puisse apprendre des chants inconnus, tel que l'on fait en préparant la sainte messe domnicale.

Il y a aussi des exercices écrits à compléter, puis à envoyer par la poste ou par voie électronique selon les modalités du cours. Bien que ces exercices ne nécéssitent pas toujours une partition entière mais seulement une partie de celle-ci, fournir la partition complète permet à l'élève d'apprendre le chant entier plus d'une manière plus profonde ou de pouvoir le faire dans un groupe s'il y a des élèves souhaitant travailler ensemble, sans ou avec un professur suffisament capable de chanter selon la méthode de Solesmes afin de guider les élèves.

Il faut toujours à l'élève une copie du manuel du niveau I. En revanche, ce livre remplace une copie soit du \textit{Graduale Romanum,} soit du \textit{Liber Usualis,} autrement appelé le « 800 » ou le \textit{Paroissien romain} dans les contextes francophones. Il ne serait pas ainsi nécessaire de trouver chaque partition par les moyens électroniques. Toutefois, l'élève doit encore écrire ses propres tables et de copier ou de reproduire des partitions selon les consignes des certains exercices écrits qui se trouvent dans le manuel, ce livre ne contenant que les partitions.
\end{otherlanguage} 

 \clearpage\thispagestyle{empty}